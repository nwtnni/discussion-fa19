\documentclass{beamer}
\usetheme{Szeged}
\usecolortheme{beaver}

\usepackage{minted}
\usemintedstyle{pastie}

\usepackage{graphicx}
\usepackage{hyperref}
\hypersetup{
    colorlinks=true,
    linkcolor=blue,
    filecolor=magenta,      
    urlcolor=cyan,
}

\title{Discussion 20}
\subtitle{Type Checking}
\author{Kenneth Fang (kwf37), Newton Ni (cn279)}



\begin{document}

    \begin{frame}
        \titlepage{}
    \end{frame}
    
    \begin{frame}{Agenda}
    \begin{enumerate}
        \item Preview Type Checking Relation
        \item Exercises
        \item Midterm Projects
    \end{enumerate}
    \end{frame}
    
    \begin{frame}[fragile]{Type Checking Relation}
    The type checking relation defines what programs are ``good" and what programs are ``bad" \pause 
        \begin{itemize}
        \item We say $e$ is well-typed if
        \begin{verbatim}
            T |- e : t
        \end{verbatim} \pause
            \item $T$ is the typing context (sometimes called $\Gamma$)
                \begin{itemize}
                    \item It is a map from variable names $\to$ types
                    \item A lot like the environment we saw when implementing an environment-model interpreter
                \end{itemize} \pause
            \item $e$ is the expression pause
            \item $t$ is the type of the expression (sometimes called $\tau$) \pause 
            \item Read as ``Expression e has type t under context T"
        \end{itemize}
    \end{frame}
    
    \begin{frame}[fragile]{Static Semantics: Integer Addition}
    Suppose we have a Bool Type and an Int Type- here's how we can define the type relation for addition:
    \begin{columns}
        \begin{column}{0.48\textwidth}
\begin{verbatim}
T |- e1 + e2 : int
     if  ???
\end{verbatim}
    Static Semantics
        \end{column}
        \begin{column}{0.48\textwidth}
        \end{column}
    \end{columns}
    \end{frame}
    
    \begin{frame}[fragile]{Static Semantics: Integer Addition}
    Suppose we have a Bool Type and an Int Type- here's how we can define the type relation for addition:
    \begin{columns}
        \begin{column}{0.48\textwidth}
\begin{verbatim}
T |- e1 + e2 : int
     if  T |- e1 : int
     and T |- e2 : int
\end{verbatim}
    Static Semantics
        \end{column}
        \begin{column}{0.48\textwidth}
\begin{verbatim}
<env, e1 + e2> => v
  if <env, e1> => v1
  and <env, e2> => v2
  and v1 + v2 = i
\end{verbatim}
    Dynamic Semantics (Environment Model)
        \end{column}
    \end{columns}
    \end{frame}
    
    \begin{frame}[fragile]{Static Semantics: Let Expressions}
    Suppose we have a Bool Type and an Int Type- here's how we can define the type relation for let expressions:
    \begin{columns}
        \begin{column}{0.48\textwidth}
\begin{verbatim}
T |- let x = e1 in e2 : t
  if ???
\end{verbatim}
    Static Semantics
        \end{column}
        \begin{column}{0.48\textwidth}
\begin{verbatim}
<env, let x = e1 in e2> => v
  if <env, e1> -->* v1
  and <env[x->v1], e2> => v
\end{verbatim}
    Dynamic Semantics (Environment Model)
        \end{column}
    \end{columns}
    \end{frame}
    
    \begin{frame}[fragile]{Static Semantics: Let Expressions}
    Suppose we have a Bool Type and an Int Type- here's how we can define the type relation for let expressions:
    \begin{columns}
        \begin{column}{0.48\textwidth}
\begin{verbatim}
T |- let x = e1 in e2 : t
  if T |- e1 : t1
  and T[x->t1] |- e2 : t
\end{verbatim}
    Static Semantics
        \end{column}
        \begin{column}{0.48\textwidth}
\begin{verbatim}
<env, let x = e1 in e2> => v
  if <env, e1> -->* v1
  and <env[x->v1], e2> => v
\end{verbatim}
    Dynamic Semantics (Environment Model)
        \end{column}
    \end{columns}
    \end{frame}
    
    \begin{frame}[fragile]{Static Semantics: If-Then-Else}
    Suppose we have a Bool Type and an Int Type- here's how we can define the type relation for if statements:
\begin{verbatim}
T |- if e1 then e2 else e3 : t
  if ???
\end{verbatim}
    \end{frame}
    
    \begin{frame}[fragile]{Static Semantics: If-Then-Else}
    Suppose we have a Bool Type and an Int Type- here's how we can define the type relation for if statements:
\begin{verbatim}
T |- if e1 then e2 else e3 : t
  if T |- e1 : bool
  and e2 : t
  and e2 : t
\end{verbatim}
    \end{frame}
    
    \begin{frame}{Type Soundness}
        What does it mean for a program to be good? \pause \\
        
        Usually we want these two super-useful properties:
        \begin{itemize}
            \item \textbf{Progress:} if \textbf{e:t}, then \textbf{e} is a value or can take a step
            \item \textbf{Preservation:} if \textbf{e:t} and \textbf{e}$\to$ \textbf{e'}, then \textbf{e':t}
        \end{itemize}
    \end{frame}
    
    \begin{frame}{Type Soundness}
        What does it mean for a program to be good? \\
        
        Usually we want these two super-useful properties:
        \begin{itemize}
            \item \textbf{Progress:} Well-typed programs always run to completion
            \item \textbf{Preservation:} Evaluation does not change the type of an expression
        \end{itemize}
    \end{frame}
    
    \begin{frame}{Type Soundness}
        What does it mean for a program to be good? \\
        
        Usually we want these two super-useful properties:
        \begin{itemize}
            \item \textbf{Progress:} Well-typed programs always run to completion
            \item \textbf{Preservation:} Evaluation does not change the type of an expression
        \end{itemize} \pause
        
        If these two properties hold for a type system, we say that type system is ``sound"
    \end{frame}
    
    \begin{frame}[fragile]{Type Soundness: Example}
        Here's an unsound example: 
        \begin{verbatim}
        T |- if e1 then e2 else e3 : t2
             if  T |- e1 : bool
             and T |- e2 : t2
             and T |- e3 : t3
        \end{verbatim}
        Does this violate Progress or Preservation (or neither)?
        
        
        \begin{itemize}
            \item \textbf{Progress:} Well-typed programs always run to completion
            \item \textbf{Preservation:} Evaluation does not change the type of an expression
        \end{itemize}
    \end{frame}

\end{document}

