\documentclass{beamer}
\usetheme{Szeged}
\usecolortheme{beaver}

\usepackage{minted}
\usemintedstyle{pastie}
\newcommand{\code}[1]{\mintinline{ocaml}{#1}}

\title{Discussion 15}
\subtitle{Promises}
\author{Kenneth Fang (kwf37), Newton Ni (cn279)}
\date{March 20, 2019}

\begin{document}

    \begin{frame}
        \titlepage{}
    \end{frame}

    \begin{frame}{Key Ideas}
        \begin{itemize}
            \item<1-> Like laziness: do it \textbf{eventually}
            \item<2-> Bind is your best friend: \code{'a t -> ('a -> 'b t) -> 'b t}
        \end{itemize}
    \end{frame}

    \begin{frame}{Why Promises?}
        \begin{itemize}
            \item<1-> Very useful for IO-bound computations and message-passing
            \item<2-> Don't just sit around waiting
            \item<3-> Push vs. poll architecture
            \item<4-> Not useful by themselves: need OS support
        \end{itemize}
    \end{frame}

    \begin{frame}{Bind}
        \begin{itemize}
            \item<1-> Pipelining should be familiar by now: \code{|>}
            \item<2-> \code{(|>) : 'a -> ('a -> 'b) -> 'b}
            \item<3-> Bind is just a fancy pipeline
            \item<4-> \code{(>>=) : 'a t -> ('a -> 'b t) -> 'b t}
            \item<5-> Both abstract over \textbf{sequential} composition
        \end{itemize}
    \end{frame}

    \begin{frame}[fragile=singleslide]{Bind: Function Syntax}
        Does this remind you of anything?

        \begin{minted}[autogobble]{ocaml}
            bind promise_a (
                fun a -> bind promise_b (
                    fun b -> bind promise_c (
                        fun c -> do_something_with a b c
                    )
                )
            )
        \end{minted}
    \end{frame}

    \begin{frame}[fragile=singleslide]{Application: Function Syntax}
        \begin{minted}[autogobble]{ocaml}
        map_refl refl (
            map_rotors_r_to_l rotors (
                map_plug plugboard (
                    index c
                )
            )
        )
        \end{minted}
    \end{frame}

    \begin{frame}[fragile=singleslide]{Application: Operator Syntax}
        \begin{minted}[autogobble]{ocaml}
            c |> index
              |> map_plug plugboard
              |> map_rotors_r_to_l rotors
              |> map_refl refl
        \end{minted}
    \end{frame}

    \begin{frame}[fragile=singleslide]{Bind: Operator Syntax}
        \begin{minted}[autogobble]{ocaml}
            promise_a >>= fun a ->
            promise_b >>= fun b ->
            promise_c >>= fun c ->
            do_something_with a b c
        \end{minted}
    \end{frame}

    \begin{frame}{Bind: PPX Extensions}
        \begin{itemize}
            \item<1-> We don't see PPX extensions in this course for the most part
            \item<2-> Compile-time code generation!
            \item<3-> Textbook section 8.20 for the curious
            \item<4-> Sneak peek: \code{let/lwt a = promise_a in let/lwt b = promise_b in do_something a b}
        \end{itemize}
    \end{frame}

    \begin{frame}{Libraries}
        \begin{itemize}
            \item<1-> \code{#require "lwt";;} - core modules and types
            \item<2-> \code{#require "lwt.unix";;} - Unix bindings for socket/file IO
            \item<3-> \code{UTop.set_auto_run_lwt false;;}
            \item<3-> Special utop behavior for top-level promises (see textbook 8.19)
        \end{itemize}
    \end{frame}

    \begin{frame}[fragile=singleslide]{Examples: Sequential Composition}
        \begin{minted}[autogobble]{ocaml}
            let a () = Lwt_unix.sleep 2.0;;
            let b () = Lwt_unix.sleep 2.0;;
            Lwt_main.run begin a () >>= b end;;
        \end{minted}
    \end{frame}

    \begin{frame}[fragile=singleslide]{Examples: Sequential Composition(?)}
        \begin{minted}[autogobble]{ocaml}
            let a = Lwt_unix.sleep 2.0;;
            let b = Lwt_unix.sleep 2.0;;
            Lwt_main.run begin a >>= fun () -> b end;;
        \end{minted}
    \end{frame}

    \begin{frame}[fragile=singleslide]{Examples: Concurrent Composition}
        \begin{minted}[autogobble]{ocaml}  
            Lwt_main.run begin
              let a = Lwt_unix.sleep 2.0 in
              let b = Lwt_unix.sleep 2.0 in
              a >>= fun () -> b
            end;;
        \end{minted}
    \end{frame}

    \begin{frame}[fragile=singleslide]{Examples: Concurrent Composition}
        \begin{minted}[autogobble]{ocaml}  
            Lwt_main.run begin
              let a = Lwt_unix.sleep 2.0 in
              let b = Lwt_unix.sleep 2.0 in
              Lwt.choose [a; b]
            end;;
        \end{minted}
    \end{frame}

    \begin{frame}{More Examples}
        \begin{itemize}
            \item<1-> \code{delay.ml} - try and predict the executions!
            \item<2-> \code{crawl.ml} - interactive "game" loop
            \item<3-> \code{client.ml} and \code{server.ml} - working chat server!
        \end{itemize}
    \end{frame}


\end{document}
