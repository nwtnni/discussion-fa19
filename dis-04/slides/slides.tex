\documentclass{beamer}
\usetheme{Szeged}
\usecolortheme{beaver}

\usepackage{minted}
\usemintedstyle{pastie}

\usepackage[percent]{overpic}

\newcommand{\code}[1]{\mintinline{ocaml}{#1}}

\title{Discussion 04}
\subtitle{Variants}
\author{Kenneth Fang (kwf37), Newton Ni (cn279)}
\date{Feb. 6, 2019}

\begin{document}

    \begin{frame}
        \titlepage{}
    \end{frame}

    \begin{frame}{Key Concepts}
        \begin{itemize}
            \item<1-> Sum types are \textbf{fundamental}.
            \item<2-> Option types enable \textbf{static analysis}.
        \end{itemize}
    \end{frame}

    \begin{frame}
        \centering
        \begin{overpic}[width=0.75\textwidth]{brain.jpg}
            \put(3, 94) {\code{type coin =}}
            \put(3, 89) {\code{| Penny}}
            \put(3, 84) {\code{| Nickel}}
            \put(3, 79) {\code{| Dime}}
            \put(3, 74) {\code{| Quarter}}

            \put(3, 60) {\code{type shape =}}
            \put(3, 55) {\code{| Point }}
            \put(3, 50) {\code{| Square of int }}
            \put(3, 45) {\code{| Circle of }}
            \put(3, 40) {\code{{ radius: int }}}

            \put(3, 25) {\code{type 'a t =}}
            \put(3, 20) {\code{| Leaf}}
            \put(3, 15) {\code{| Node of}}
            \put(3, 10) {\code{'a * 'a t * 'a t}}
        \end{overpic}
    \end{frame}

    \begin{frame}[fragile=singleslide]{Variants: Enumerations}
        \centering
        \begin{minted}{ocaml}
            type coin =
            | Penny
            | Nickel
            | Dime
            | Quarter
        \end{minted}
    \end{frame}

    \begin{frame}{Variants: Enumerations}
        Useful when $\ldots$
        \begin{itemize}
            \item<1-> You have a small number of constants
            \item<2-> e.g.\ card suits, keyboard buttons, Crayon colors
        \end{itemize}
    \end{frame}

    \begin{frame}[fragile=singleslide]{Variants: Tagged Unions}
        \begin{minted}{ocaml}
            type shape =
            | Point
            | Square of int
            | Circle of { radius: int }
        \end{minted}
    \end{frame}

    \begin{frame}
        Useful when $\ldots$
        \begin{itemize}
            \item<1-> You have different representations of a concept
            \item<2-> e.g. state machines, errors, class hierarchies
        \end{itemize}

    \end{frame}

    \begin{frame}[fragile=singleslide]{Variants: Recursion and Polymorphism}
        \begin{minted}{ocaml}
            type 'a tree =
            | Leaf
            | Node of 'a * 'a tree * 'a tree
        \end{minted}
    \end{frame}

    \begin{frame}{Variants: Recursion and Polymorphism}
        Useful when $\ldots$
        \begin{itemize}
            \item<1-> You have inductively defined (self-similar) data
            \item<2-> e.g. naturals, trees, languages, games
        \end{itemize}
    \end{frame}
    
    \begin{frame}[fragile=singleslide]{Exercise: Calculator}
        \begin{minted}{ocaml} 
        (** Represents a binary operator. *)
        type bin =
        | Add
        | Sub
        | Mul
        | Div

        (** Represents an expression. *)
        type exp =
        | Int of int
        | Bin of exp * bin * exp
        \end{minted}
    \end{frame}

    \begin{frame}{Exercise: Calculator}
        \begin{itemize}
            \item<1-> You only have to worry about \texttt{calculator.ml}
            \item<2-> Three increasingly interesting functions:
            \item<3-> \code{let rec string_of_bin (b: bin) : string}
            \item<4-> \code{let rec string_of_exp (e: exp) : string}
            \item<5-> \code{let rec eval (e: exp) : (int option)}
        \end{itemize}
    \end{frame}

    \begin{frame}{Exercise: Calculator}
        \begin{itemize}
            \item<1-> Use \texttt{make} to compile and run the calculator
            \item<2-> Calculator will accept expressions in REPL (e.g. (5 * (2 - 3)))
        \end{itemize}
    \end{frame}

    \begin{frame}{Recitation Exercises}
    \end{frame}

\end{document}
