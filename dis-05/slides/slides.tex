\documentclass{beamer}
\usetheme{Szeged}
\usecolortheme{beaver}

\usepackage{minted}
\usemintedstyle{pastie}

\usepackage{graphicx}
\usepackage{hyperref}
\hypersetup{
    colorlinks=true,
    linkcolor=blue,
    filecolor=magenta,      
    urlcolor=cyan,
}

\title{Discussion 05}
\subtitle{Higher-Order Functions}
\author{Kenneth Fang (kwf37), Newton Ni (cn279)}
\date{Feb. 11, 2019}

\begin{document}

    \begin{frame}
        \titlepage{}
    \end{frame}
    
    \begin{frame}{Agenda}
    \begin{enumerate}
        \item Review Options (optional) ((pun intended))
        
        \item Higher-Order Functions
        
        \item Recitation 6 (Please pull it up as you enter)
    \end{enumerate}
    \end{frame}
    
    \begin{frame}[fragile]{(Review) Options}
        Options are a common variant type built into the language:
        \begin{itemize}
            \item \mintinline{ocaml}{type 'a option = Some of 'a | None} \pause
            
            \item This is an example of ``parametric polymorphism." The \mintinline{ocaml}{option} type is paramaterized (hence parametric) on type \mintinline{ocaml}{'a}, which can be any type (hence polymorphic). \pause
            
            \item Example:
            \begin{minted}[autogobble]{ocaml}
            let safe_division (a:int) (b:int) : int option =
              if b = 0 then None
              else Some (a / b)
            \end{minted}
        \end{itemize}
    \end{frame}
    
    \begin{frame}[fragile]{(Review) Option Examples}
    \begin{itemize}
        \item 
        \begin{minted}[autogobble]{ocaml}
        let safe_division (a:int) (b:int) : int option =
          if b = 0 then None
          else Some (a / b)
        \end{minted}
        \item 
            \begin{minted}[autogobble]{ocaml}
            let text_ui (a:int) (b:int) : unit =
              match safe_division a b with
              | Some x -> print_endline (string_of_int x)
              | None -> print_endline "Error: Division by zero"
            \end{minted}
    \end{itemize}
    \end{frame}
    
    \begin{frame}{Why Options?}
    \begin{itemize}
        \item My take: conceptually, options exist everywhere!
        \item In Java, any pointer could be pointing to a real value or NULL.
        \item However, using options makes this \textbf{explicit}.
        \item If a value is wrapped in an Option, you are forced to pattern match on the None case, which means you \textit{don't unexpected exceptions}.
    \end{itemize}
    \end{frame}
    
    \section{Higher-Order Functions}
    \begin{frame}[fragile]{What is a Higher-Order Function?}\pause
    \begin{itemize}
        \item Functions take in values as input arguments \pause
        \item Functions are values \pause
        \item If a function takes another function in as an input argument, it is called a ``higher-order function"
    \end{itemize}
    \end{frame}
    
    \begin{frame}[fragile]{The Simplest Example}
    \begin{minted}[autogobble]{ocaml}
    /**
     * [apply] takes the function [f] and applies it to [a].
     */
    \end{minted}
    \mintinline{ocaml}{let apply (a: 'a) (f: 'a -> 'b) : 'b = }\pause
    \mintinline{ocaml}{ f a}
    
    \pause
    \begin{itemize}
        \item [f] is a function, so [apply] is higher order!
    \end{itemize}
        
    \end{frame}
    
    \begin{frame}[fragile]{But Why Tho}
    \begin{itemize}
      \item Higher-order functions are a functional programming \textit{idiom}, or commonly used pattern in code. \pause
      \item What are some Object-Oriented programming idioms? What do they do?
      \begin{itemize} \pause
          \item Encapsulation- hides unnecessary information to make code easier to understand
          \item Inheritance- Used to share code between related classes
      \end{itemize}
    \end{itemize}
    \end{frame}
    
    \begin{frame}{Examples using Higher-Order Functions}
        
    \end{frame}
    
    \begin{frame}{Factoring Out Code}
    These are some things I think about when trying to factor code with higher-order programming
    \begin{itemize}
        \item Each function does some different kinds of \textit{computation}
        \item For example, to sum a list, your computation includes \textit{iterating} through the list and \textit{summing} the list elements.
        \item Higher-order programming can be used by abstracting the \textit{iterating} part of the computation, and passing in a function that does the \textit{summing} part.
        \item The abstracted \textit{iterating} function is the \textbf{fold} function
    \end{itemize}
        
    \end{frame}
    

\end{document}
