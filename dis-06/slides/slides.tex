\documentclass{beamer}
\usetheme{Szeged}
\usecolortheme{beaver}

\usepackage{minted}
\usemintedstyle{pastie}

\newcommand{\code}[1]{\mintinline{ocaml}{#1}}

\title{Discussion 06}
\subtitle{Modules}
\author{Kenneth Fang (kwf37), Newton Ni (cn279)}
\date{Feb. 13, 2019}

\begin{document}

    \begin{frame}
        \titlepage{}
    \end{frame}

    \begin{frame}{Key Concepts}
        \begin{itemize}
            \item<1-> Signatures are \textbf{interfaces}
            \item<2-> Abstract types enable \textbf{information hiding}
        \end{itemize}
    \end{frame}

    \begin{frame}{Signatures}

        \centering
        {\Large SIGNATURES ARE \textbf{NOT} IMPLEMENTATIONS}

    \end{frame}

    \begin{frame}[fragile=singleslide]{Signatures}
        \begin{minted}{ocaml}
        module type Zero = sig
            val zero: int
        end

        (* Q: What is [Zero.zero]? *)
        (* A: Error: Unbound module Zero *)
        \end{minted}
    \end{frame}

    \begin{frame}[fragile=singleslide]{Signatures: Undersatisfied}
        \begin{minted}{ocaml}
        module type Transition = sig
            val start: int
            val next: int -> int
        end

        module Count: Transition = struct
            let start = 0
        end
        \end{minted}
    \end{frame}

    \begin{frame}[fragile=singleslide]{Signatures: Oversatisfied}
        \begin{minted}{ocaml}
        module type Transition = sig
            val start: int
            val next: int -> int
        end

        module Count: Transition = struct
            let start = 0
            let square x = x * x
            let next x = square x
        end
        \end{minted}
    \end{frame}

    \begin{frame}{Signatures: Implicit}
        \begin{itemize}
            \item<1-> Q: do we need to explicitly write the signature?
            \item<2-> \code{module Count: Transition = struct ... end}
            \item<2-> \code{module Count = struct ... end}
            \item<3-> A: Nope---but use not clear until \textbf{functors} next week!
        \end{itemize}
    \end{frame}

    \begin{frame}{Abstract Types}
        \begin{itemize}
            \item<1-> Idiom: abstract type + module $\approx$ private data + class
            \item<2-> \code{Stack.t} vs. \code{Stack}
            \item<3-> Why private data?
            \begin{itemize}
                \item<4-> Loose coupling
                \item<5-> Invariant upholding
            \end{itemize}
        \end{itemize}
    \end{frame}

    \begin{frame}[fragile=singleslide]{Abstract Types: Loose Coupling}
        \begin{minted}{ocaml}
        module type Set = sig 
            type 'a t
            val empty: 'a t
            val insert: 'a t -> 'a -> 'a t
            val mem: 'a t -> 'a -> bool
        end

        module MySet: Set = struct
            type 'a t = 'a list
            ...
        end
        \end{minted}
    \end{frame}

    \begin{frame}[fragile=singleslide]{Abstract Types: Invariants}
        \begin{minted}{ocaml}
        module type Nonzero = sig
            type t
            val t_of_int: int -> t option
            val int_of_t: t -> int
        end

        module NonzeroSealed: Nonzero = struct
            type t = int
            let t_of_int = function
            | 0 -> None
            | n -> Some n
            let int_of_t n = n
        end
        \end{minted}
    \end{frame}

    \begin{frame}{Digression: \code{utop}}
        \begin{itemize}
            \item<1-> \texttt{\#use "source.ml"}
            \item<2-> \texttt{\#require "library"}
            \item<3-> \texttt{\#mod\_use "source.ml"}
            \item<4-> \texttt{\#load "compiled.cmo"}
            \item<5-> \texttt{\#load\_rec "compiled.cmo"}
            \item<6-> \texttt{\#trace function}
            \item<7-> \texttt{.ocamlinit}
        \end{itemize}
    \end{frame}

    \begin{frame}{Abstract Types: Experimenting}
        \begin{itemize}
            \item \code{let v: NonzeroFree.t = 0;;}
            \item \code{let v: NonzeroSealed.t = 0;;}
        \end{itemize}
    \end{frame}

\end{document}
