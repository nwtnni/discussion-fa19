\documentclass{beamer}
\usetheme{Szeged}
\usecolortheme{beaver}

\usepackage{minted}
\usemintedstyle{pastie}

\title{Discussion 14}
\subtitle{Mutability :O}
\author{Kenneth Fang (kwf37), Newton Ni (cn279)}

\begin{document}

    \begin{frame}
        \titlepage{}
    \end{frame}

    \begin{frame}{Agenda}
        \begin{enumerate}
            \item Review of Mutable Constructs
            \item Exercises
            \item Rec 14
        \end{enumerate}
    \end{frame}

    \begin{frame}[fragile=singleslide]{Mutable Fields}
        \begin{itemize}
            \item A record can have a mutable field by using the \mintinline{ocaml}{mutable} keyword.
            \item Ex:
            \begin{minted}[autogobble]{ocaml}
            type mutable_student = {
            name: string,
            mutable age: int
            }
            \end{minted}
        \end{itemize}
    \end{frame}

    \begin{frame}[fragile=singleslide]{Mutable Fields}
        Try this in utop:
        \begin{minted}[autogobble=true]{ocaml}
        type mutable_student = {
          name: string,
          mutable age: int
        }
        \end{minted}
        \begin{itemize}
            \item \mintinline{ocaml}{let kenneth = {name="kenneth";age=20}}
            \item \mintinline{ocaml}{kenneth.age <- 5}
            \item What type did the last command return?
            \item What happens if you type \mintinline{ocaml}{kenneth} now?
            \item Try running \mintinline{ocaml}{kenneth.name <- "newton"}
        \end{itemize}
    \end{frame}

    \begin{frame}[fragile=singleslide]{Refs}
        \begin{itemize}
            \item Refs are just records with a single mutable field:
            \item From lecture:
            \begin{minted}[autogobble=true]{ocaml}
            type 'a ref = { mutable contents: 'a }
            let ref x = { contents = x }
            let ( ! ) r = r.contents
            let ( := ) r newval = r.contents <- newval
            \end{minted}
            \item How can we use these?
        \end{itemize}
    \end{frame}

    \begin{frame}[fragile=singleslide]{Refs}
        Time for more utop:
        \begin{itemize}
            \item \mintinline{ocaml}{let x = ref 0}
            \item \mintinline{ocaml}{x := !x + 1}
            \item \mintinline{ocaml}{x}
            \item \mintinline{ocaml}{let y = ref 1}
            \item \mintinline{ocaml}{x = y (* What does this evaluate to? *)}
            \item \mintinline{ocaml}{x == y (* How about this? *)}
        \end{itemize}
    \end{frame}

    \begin{frame}{Physical Equality}
        Time for more utop:
        \begin{itemize}
            \item \mintinline{ocaml}{(==)} is the physical equality operator
            \pause
            \item It checks if two mutable values point to the same location in memory
            \pause
            \item Its complement is \mintinline{ocaml}{(!=)} for physically not equal
            \pause
            \item The regular equality operators \mintinline{ocaml}{(=)} and 
                \mintinline{ocaml}{(<>)} check if all the contents of each mutable 
                field are the same
            \pause
        \end{itemize}
    \end{frame}

    \begin{frame}{Physical Equality}
        Some questions: (discuss with your neighbors and raise your hand when you have an idea)
        \begin{itemize}
            \item Is it possible for \mintinline{ocaml}{(=)} to return true and \mintinline{ocaml}{(==)} to return false?
            \item Is it possible for \mintinline{ocaml}{(=)} to return false and \mintinline{ocaml}{(==)} to return true?
        \end{itemize}
    \end{frame}

    \begin{frame}[fragile=singleslide]{Physical Equality/Aliasing}
    Try out the following:
    \begin{itemize}
        \item \mintinline{ocaml}{let x = ref "wowie"}
        \item \mintinline{ocaml}{let z = x}
        \item \mintinline{ocaml}{x == z (* What does this evaluate to? *)}
        \item \mintinline{ocaml}{z := "2b || !2b"}
        \item \mintinline{ocaml}{x}
        \item \mintinline{ocaml}{z}
    \end{itemize}
    \pause

    This is an example of aliasing!
    \end{frame}

    \begin{frame}{Aliasing}
    \begin{itemize}
        \item Aliasing is when two pointers point to the same location in memory
        \pause
        \item It should be familiar from programming in Java and other imperative languages
        \pause
        \item It can be hard to keep track of!
    \end{itemize}
    \end{frame}

    \begin{frame}{Exercises}
    \begin{itemize}
        \item As usual, pull from the repo and check out the exercises
    \end{itemize}
    \end{frame}

    \begin{frame}{Rec 14}
    \end{frame}

\end{document}
