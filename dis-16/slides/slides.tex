\documentclass{beamer}
\usetheme{Szeged}
\usecolortheme{beaver}

\usepackage{minted}
\usemintedstyle{pastie}

\usepackage{graphicx}
\usepackage{hyperref}
\hypersetup{
    colorlinks=true,
    linkcolor=blue,
    filecolor=magenta,      
    urlcolor=cyan,
}

\title{Discussion 16}
\subtitle{Monads}
\author{Kenneth Fang (kwf37), Newton Ni (cn279)}



\begin{document}

    \begin{frame}
        \titlepage{}
    \end{frame}
    
    \begin{frame}{Agenda}
    \begin{enumerate}
        \item Review of Monads
        \item Examples of Monads
        \begin{itemize}
            \item Promises
            \item Options
            \item Error
            \item Lazy
            \item Non-Deterministic?
        \end{itemize}
        \item More Exercises
    \end{enumerate}
    \end{frame}
    
    \begin{frame}[fragile]{What is a Monad?}
    \begin{itemize}
        \item Monads are a \textit{design pattern} that pops up frequently in functional programming \pause
        \item A Monad is any module that satisfies the following interface:
        
        \begin{minted}[autogobble=true]{ocaml}
        module type Monad = sig
          type 'a t
          val return: 'a -> 'a t
          val bind: 'a t -> ('a -> 'b t) -> 'b t
        end
        \end{minted}
    \end{itemize}
    \end{frame}
    
    \begin{frame}[fragile]{What is a Monad?}
    \begin{itemize}
        \item Monads are a \textit{design pattern} that pops up frequently in functional programming
        
        \item A Monad is any module that satisfies the following interface:
        
        \begin{minted}[autogobble=true]{ocaml}
        module type Monad = sig
          type 'a t
          val return: 'a -> 'a t
          val (>>=): 'a t -> ('a -> 'b t) -> 'b t
        end
        \end{minted}
        
        \item We often use the infix operator \mintinline{ocaml}{(>>=)} for \mintinline{ocaml}{bind}
    \end{itemize}
    \end{frame}
    
    \begin{frame}[fragile]{What is a Monad?}
        
        \begin{minted}[autogobble=true]{ocaml}
        module type Monad = sig
          type 'a t
          val return: 'a -> 'a t
          val (>>=): 'a t -> ('a -> 'b t) -> 'b t
        end
        \end{minted}
    \begin{itemize}
        \item The \mintinline{ocaml}{return} function lets a user lift a value into the Monad type \pause
        \item The \mintinline{ocaml}{(>>=)} function does computation on values that are wrapped in the Monad type
    \end{itemize}
    \end{frame}
    
    
    \begin{frame}[fragile]{Some Intuition}
    \begin{itemize}
        \item The type \mintinline{ocaml}{t} adds some extra computation to your functions \pause
        \item It wraps around a piece of data and gives it the ability to do more \pause
        \item The extra computation is done \textit{implicitly} \pause
        \item Compare \mintinline{ocaml}{(>>=)} (bind) to \mintinline{ocaml}{(|>)} (pipeline):
        
        
        \begin{minted}[autogobble=true]{ocaml}
          (|>): 'a -> ('a -> 'b) -> 'b
          (>>=): 'a t -> ('a -> 'b t) -> 'b t
        \end{minted}
    \end{itemize}
    \end{frame}
    
    
    \begin{frame}[fragile]{Example: Optional Monad}
        Open up today's exercises and try to implement this with your groupmates:
        
        \begin{minted}[autogobble=true]{ocaml}
        module Optional = struct
            type 'a t = 'a option
            let return a = failwith "Unimplemented"
            let (>>=) a f = failwith "Unimplemented"
        end
        \end{minted}
    \end{frame}
    
    
    \begin{frame}[fragile]{Example: Optional Monad}
        Open up today's exercises and try to implement this with your groupmates:
        
        \begin{minted}[autogobble=true]{ocaml}
        module Optional = struct
            type 'a t = 'a option
            let return a = Some a
            let (>>=) a f =
              match a with
              | Some a -> f a
              | None -> None
        end
        \end{minted}
    \end{frame}
    
    
    \begin{frame}[fragile]{Example: Error Monad}
        Open up today's exercises and try to implement this with your groupmates:
        
        \begin{minted}[autogobble=true]{ocaml}
        module Error = struct
            type 'a t = Error of string | Val of 'a
            let return a = failwith "Unimplemented"
            let (>>=) a f = failwith "Unimplemented"
        end
        \end{minted}
    \end{frame}
    
    
    \begin{frame}[fragile]{Example: Error Monad}
        Open up today's exercises and try to implement this with your groupmates:
        
        \begin{minted}[autogobble=true]{ocaml}
        module Error = struct
            type 'a t = Error of string | Val of 'a
            let return a = Val a
            let (>>=) a f =
              match a with
              | Val a -> f a
              | Error s -> print_endline s; Error s
        end
        \end{minted}
    \end{frame}
    
    
    \begin{frame}[fragile]{Example: Error Monad}
        Open up today's exercises and try to implement this with your groupmates:
        
        \begin{minted}[autogobble=true]{ocaml}
        module Error = struct
            type 'a t = Error | Val of 'a
            let return a = failwith "Unimplemented"
            let (>>=) a f = failwith "Unimplemented"
        end
        \end{minted}
    \end{frame}
    
    
    
    \begin{frame}[fragile]{Example: Lazy Monad}
        Open up today's exercises and try to implement this with your groupmates:
        
        \begin{minted}[autogobble=true]{ocaml}
        type 'a t = unit -> 'a
        let return a = failwith "Unimplemented"
        let (>>=) a f = failwith "Unimplemented"
    
        (** An extra function specific to Lazy
            [force l] forces the computation on l,
            immediately evaluating it to a value *)
        let force (l: 'a t): 'a = failwith "Unimplemented" 
        \end{minted}
    \end{frame}
    
    \begin{frame}[fragile]{Example: Lazy Monad}
        Open up today's exercises and try to implement this with your groupmates:
        
        \begin{minted}[autogobble=true]{ocaml}
        type 'a t = unit -> 'a
        let return a = failwith "Unimplemented"
        let (>>=) a f = failwith "Unimplemented"
    
        (** An extra function specific to Lazy
            [force l] forces the computation on l,
            immediately evaluating it to a value *)
        let force (l: 'a t): 'a = failwith "Unimplemented" 
        \end{minted}
    \end{frame}
    
    
    \begin{frame}[fragile]{Example: Lazy Monad}
        Open up today's exercises and try to implement this with your groupmates:
        
        \begin{minted}[autogobble=true]{ocaml}
        type 'a t = unit -> 'a
        let return a = fun () -> a
        let (>>=) a f = f (a ())
    
        (** An extra function specific to Lazy
            [force l] forces the computation on l,
            immediately evaluating it to a value *)
        let force (l: 'a t): 'a = l ()
        \end{minted}
    \end{frame}
    
    
    
    \begin{frame}[fragile]{Example: Nondeterministic Monad???}
        Open up today's exercises and try to implement this with your groupmates:
        
        \begin{minted}[autogobble=true]{ocaml}

        module NonDeterministic = struct
            type 'a t = 'a list
            let return a = failwith "Unimplemented"
            let (>>=) a f = failwith "Unimplemented"
        end
        \end{minted}
    \end{frame}
    
    
    
    \begin{frame}[fragile]{Example: Nondeterministic Monad???}
        Open up today's exercises and try to implement this with your groupmates:
        
        \begin{minted}[autogobble=true]{ocaml}

        module NonDeterministic = struct
            type 'a t = 'a list
            let return a = [a]
            let (>>=) a f = f (choose_random a)
        end
        \end{minted}
    \end{frame}
    
    
    
    
    

\end{document}

